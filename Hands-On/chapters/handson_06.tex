% ============================================================
%  Hands-On 6 — Otomatisasi Proses Build dengan Makefile
% ============================================================

\chapter{Otomatisasi Proses Build dengan Makefile}
\label{chap:handson-06}

% --- 1. Tugas ---
\section{Tugas}

Jawablah pertanyaan-pertanyaan berikut berdasarkan hasil praktikum Anda.
Sertakan bukti berupa \textit{screenshot} atau penggalan keluaran terminal
yang relevan pada setiap jawaban.

\begin{enumerate}
    \item \textbf{[Topik Pertanyaan 1]}\\
    \textit{TODO: ganti dengan pertanyaan sesungguhnya.}\\[0.3cm]
    \textbf{Jawaban:}\\
    \textit{[Isi jawaban di sini]}

    \item \textbf{[Topik Pertanyaan 2]}\\
    \textit{TODO: ganti dengan pertanyaan sesungguhnya.}
    \begin{enumerate}
        \item[(a)] Sub-pertanyaan a.\\
        \textbf{Jawaban:} \textit{[Isi jawaban di sini]}
        \item[(b)] Sub-pertanyaan b.\\
        \textbf{Jawaban:} \textit{[Isi jawaban di sini]}
    \end{enumerate}

    \item \textbf{[Topik Pertanyaan 3]}\\
    \textit{TODO: ganti dengan pertanyaan sesungguhnya.}\\[0.3cm]
    \textbf{Jawaban:}\\
    \textit{[Isi jawaban di sini]}

    \item \textbf{[Eksplorasi Mandiri]}\\
    \textit{TODO: ganti dengan pertanyaan eksplorasi sesungguhnya.}\\[0.3cm]
    \textbf{Jawaban:}\\
    \textit{[Isi jawaban di sini]}
\end{enumerate}

% --- 2. Kesimpulan ---
\newpage
\section{Kesimpulan}

\noindent\textit{[TODO: Tuliskan kesimpulan dari praktikum yang telah dilaksanakan.
Kesimpulan hendaknya menjawab tujuan-tujuan pada modul praktikum, minimal 2--3 paragraf.]}

% --- 3. Lampiran: Bukti Penggunaan AI ---
\newpage
\section{Lampiran: Bukti Penggunaan AI}
\label{sec:h06-lampiran-ai}

Sesuai dengan kebijakan penggunaan kecerdasan buatan (AI) dalam kegiatan akademik,
praktikan \textbf{wajib} mencantumkan setiap penggunaan alat bantu AI yang dilakukan
selama pengerjaan hands-on ini. Ketidakjujuran dalam pelaporan penggunaan AI dianggap
sebagai pelanggaran integritas akademik.

\subsection{Identitas Penggunaan AI}

\begin{table}[h]
\caption{Informasi Alat AI yang Digunakan}
\label{tab:h06-ai-info}
\centering
\begin{tabulary}{\textwidth}{|L|L|}
\hline
\textbf{Keterangan}   & \textbf{Isian} \\ \hline
Nama alat AI          & \textit{(contoh: ChatGPT, GitHub Copilot, Gemini, dst.)} \\ \hline
Versi / Model         & \textit{(contoh: GPT-4o, Gemini 1.5 Pro, dst.)} \\ \hline
Tanggal penggunaan    & \textit{TODO: isi tanggal} \\ \hline
Tujuan penggunaan     & \textit{TODO: jelaskan secara singkat untuk apa AI digunakan} \\ \hline
\end{tabulary}
\end{table}

\subsection{Tangkapan Layar Percakapan AI}

Sertakan tangkapan layar (\textit{screenshot}) percakapan dengan AI sebagai
bukti otentik. Setiap gambar harus memperlihatkan \textbf{prompt} yang dikirimkan
dan \textbf{respons} yang diterima secara jelas.

\begin{figure}[h]
    \centering
    % TODO: ganti dengan screenshot percakapan AI Anda
    \includegraphics[width=0.85\textwidth]{Figure/ifitera-header.png}
    \caption{Tangkapan Layar Percakapan AI --- Sesi 1 (TODO: ganti caption)}
    \label{fig:h06-ai-screenshot-1}
\end{figure}
