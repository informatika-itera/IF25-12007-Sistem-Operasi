% ============================================================
%  BAB 1 - Tujuan dan Output Praktikum
%  TODO: Ganti placeholder dengan konten sesungguhnya.
% ============================================================

\section{Tujuan dan \textit{Output} Praktikum}

\subsection{Tujuan Praktikum}
Setelah menyelesaikan praktikum ini, mahasiswa diharapkan mampu:
\begin{enumerate}
    \item Menjelaskan apa itu sistem operasi dalam bahasa sederhana.
    \item Mengidentifikasi peran sistem operasi dalam kehidupan sehari-hari.
    \item Membedakan antarmuka \textit{GUI (Graphical User Interface)} dan \textit{CLI (Command Line Interface)} berdasarkan pengalaman penggunaan.
    \item Memahami konsep dasar \textit{virtual machine} secara konseptual.
    \item Menjelaskan mengapa \textit{virtual machine} digunakan dalam praktikum sistem operasi.
\end{enumerate}


\subsection{\textit{Output} Praktikum}
Pada akhir praktikum ini, mahasiswa diharapkan menghasilkan:
\begin{enumerate}
    \item Pemahaman tentang definisi dan fungsi sistem operasi sebagai perangkat lunak yang mengelola sumber daya komputer.
    \item Pengetahuan tentang perbedaan antara \textit{GUI} dan \textit{CLI}, termasuk kelebihan dan kekurangan masing-masing.
    \item Pemahaman konsep \textit{virtual machine} sebagai simulasi perangkat keras yang memungkinkan menjalankan beberapa sistem operasi pada satu mesin fisik.
\end{enumerate}
