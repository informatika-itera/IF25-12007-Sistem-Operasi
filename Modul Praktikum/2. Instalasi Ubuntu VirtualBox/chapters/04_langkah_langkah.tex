% ============================================================
%  BAB 4 - Langkah-Langkah Praktikum
% ============================================================

\newpage
\section{Langkah-Langkah Praktikum}

\begin{enumerate}
    \item \textbf{Instalasi VirtualBox}

    Langkah pertama yang harus dilakukan adalah menginstal software virtualisasi, yaitu Oracle VM VirtualBox.

    Unduh VirtualBox melalui tautan resmi berikut:

    \url{https://www.virtualbox.org/wiki/Downloads}

    \item \textbf{Mengunduh File ISO Ubuntu Desktop}

    Setelah VirtualBox berhasil diinstal, langkah berikutnya adalah menyiapkan sistem operasi yang akan dijalankan di dalam Virtual Machine, yaitu Ubuntu Desktop.

    Unduh file ISO Ubuntu Desktop melalui tautan resmi berikut:

    \url{https://ubuntu.com/download/desktop}

    \item \textbf{Setup Virtual Machine}

    Setelah VirtualBox terinstal dan file ISO Ubuntu sudah diunduh, langkah selanjutnya adalah melakukan setup Virtual Machine di VirtualBox.

    \begin{enumerate}
        \item Buka aplikasi Oracle VM VirtualBox, kemudian klik tombol \textbf{New}.

        \begin{figure}[H]
            \centering
            \includegraphics[width=0.75\textwidth]{Figure/kliknew.png}
            \caption{Tampilan Oracle VM VirtualBox saat memilih tombol \textbf{New}.}
            \label{fig:virtualbox-klik-new}
        \end{figure}
        \newpage
        \item Pada bagian \textbf{Name}, isi nama mesin virtual (misalnya \texttt{Ubuntu-Elsa}). Pada bagian \textbf{ISO Image}, pilih file ISO Ubuntu yang sudah diunduh, lalu klik \textbf{Next}. 

        \begin{figure}[H]
            \centering
            \includegraphics[width=0.75\textwidth]{Figure/gambar2.png}
            \caption{Tampilan Oracle VM VirtualBox saat memilih nama mesin virtual dan file ISO Ubuntu.}
            \label{fig:virtualbox-name-iso}
        \end{figure}

        \item Pada menu \textbf{Unattended Install}, isi \textbf{Username}, \textbf{Password}, dan \textbf{Hostname} (tanpa spasi), kemudian klik \textbf{Next}. 

        \begin{figure}[H]
            \centering
            \includegraphics[width=0.75\textwidth]{Figure/gambar3.png}
            \caption{Tampilan Oracle VM VirtualBox pada tahap \textbf{Unattended Install}.}
            \label{fig:virtualbox-unattended-install}
        \end{figure}

        \item Pada menu \textbf{Hardware}, atur \textbf{Base Memory} minimal 4096 MB dan \textbf{Processor} minimal 2 core (sesuaikan dengan spesifikasi perangkat), lalu klik \textbf{Finish}. 

        \begin{figure}[H]
            \centering
            \includegraphics[width=0.75\textwidth]{Figure/gambar4.png}
            \caption{Tampilan pengaturan \textbf{Hardware} pada Oracle VM VirtualBox (Base Memory dan Processor).}
            \label{fig:virtualbox-hardware}
        \end{figure}

        \item Jalankan mesin virtual yang sudah dibuat, lalu pada tampilan awal instalasi pilih opsi \textbf{"Try or Install Ubuntu"}.

        \begin{figure}[H]
            \centering
            \includegraphics[width=0.75\textwidth]{Figure/gambar5.png}
            \caption{Tampilan awal instalasi Ubuntu saat memilih opsi \textbf{"Try or Install Ubuntu"}.}
            \label{fig:virtualbox-try-install}
        \end{figure}

        \item Lanjutkan proses instalasi sampai tahap \textit{copying files} selesai. Proses ini biasanya memerlukan waktu beberapa puluh menit, kemudian lakukan restart VM jika diminta. 
\end{enumerate}





\end{enumerate}

