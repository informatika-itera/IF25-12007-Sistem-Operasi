% ============================================================
%  Modul 8 — Implementasi Multi-threading
% ============================================================

\chapter{Implementasi Multi-threading}
\label{chap:modul-08}

% --- 1. Tujuan dan Output Praktikum ---
\section{Tujuan dan Output Praktikum}

\subsection{Tujuan Praktikum}
Setelah menyelesaikan praktikum ini, mahasiswa diharapkan mampu:
\begin{enumerate}
    \item Tujuan praktikum pertama.
    \item Tujuan praktikum kedua.
    \item Tujuan praktikum ketiga.
    \item Tujuan praktikum keempat.
\end{enumerate}

\subsection{Output Praktikum}
Pada akhir praktikum ini, mahasiswa diharapkan menghasilkan:
\begin{itemize}
    \item Output pertama yang diharapkan.
    \item Output kedua yang diharapkan.
    \item Output ketiga yang diharapkan.
\end{itemize}

% --- 2. Dasar Teori ---
\newpage
\section{Dasar Teori}

\subsection{Sub-Teori Pertama}
\lipsum[2]

\subsection{Sub-Teori Kedua}
\lipsum[3]

\subsubsection{Sub-Sub-Teori A}
\lipsum[4]

\subsubsection{Sub-Sub-Teori B}
\lipsum[5]

\subsection{Sub-Teori Ketiga}
\lipsum[6]

Tabel \ref{tab:m08-dasar-teori} merangkum hal-hal penting terkait teori di atas.

\begin{table}[h]
\caption{Tabel Ringkasan Dasar Teori}
\label{tab:m08-dasar-teori}
\centering
\begin{tabulary}{\textwidth}{|L|L|}
\hline
\textbf{Konsep} & \textbf{Penjelasan Singkat} \\ \hline
Konsep A        & \textit{TODO: isi penjelasan} \\ \hline
Konsep B        & \textit{TODO: isi penjelasan} \\ \hline
Konsep C        & \textit{TODO: isi penjelasan} \\ \hline
Konsep D        & \textit{TODO: isi penjelasan} \\ \hline
\end{tabulary}
\end{table}

% --- 3. Alat dan Bahan ---
\newpage
\section{Alat dan Bahan}
\label{sec:m08-alat-bahan}

\subsection{Perangkat Keras}
\begin{itemize}
    \item Spesifikasi perangkat keras pertama.
    \item Spesifikasi perangkat keras kedua.
\end{itemize}

\subsection{Perangkat Lunak}
\begin{itemize}
    \item Perangkat lunak pertama beserta versinya.
    \item Perangkat lunak kedua beserta versinya.
    \item Perangkat lunak ketiga beserta versinya.
\end{itemize}

% --- 4. Langkah-Langkah Praktikum ---
\newpage
\section{Langkah-Langkah Praktikum}

\subsection{Persiapan Lingkungan}
\label{sec:m08-persiapan}
\lipsum[7]

\begin{enumerate}
    \item Langkah persiapan pertama.
    \item Langkah persiapan kedua.
    \begin{lstlisting}[language=bash, caption={Perintah verifikasi lingkungan}]
# TODO: ganti dengan perintah sesungguhnya
echo "Lingkungan siap"
    \end{lstlisting}
\end{enumerate}

\subsection{Sesi 1: Nama Sesi Pertama}
\label{sec:m08-sesi-1}
\lipsum[8]

\begin{enumerate}
    \item Langkah pertama sesi ini.
    \begin{lstlisting}[language=bash, caption={Perintah sesi pertama}]
# TODO: ganti dengan perintah sesungguhnya
perintah --opsi argumen
    \end{lstlisting}

    \item Langkah kedua sesi ini.
    \begin{lstlisting}[language=bash, caption={Perintah lanjutan}]
# TODO: ganti dengan perintah sesungguhnya
perintah_lanjutan
    \end{lstlisting}
\end{enumerate}

\begin{figure}[h]
    \centering
    % TODO: ganti dengan screenshot hasil praktikum
    \includegraphics[width=0.75\textwidth]{Figure/ifitera-header.png}
    \caption{Hasil Sesi 1 --- TODO: ganti caption}
    \label{fig:m08-sesi1}
\end{figure}

\subsection{Sesi 2: Nama Sesi Kedua}
\label{sec:m08-sesi-2}
\lipsum[9]

\begin{enumerate}
    \item Langkah pertama sesi kedua.
    \item Langkah kedua sesi kedua.
    \begin{lstlisting}[language=bash, caption={Perintah sesi kedua}]
# TODO: ganti dengan perintah sesungguhnya
perintah_sesi_dua
    \end{lstlisting}
    \item Langkah ketiga sesi kedua.
\end{enumerate}

\begin{figure}[h]
    \centering
    % TODO: ganti dengan screenshot hasil praktikum
    \includegraphics[width=0.75\textwidth]{Figure/ifitera-header.png}
    \caption{Hasil Sesi 2 --- TODO: ganti caption}
    \label{fig:m08-sesi2}
\end{figure}
