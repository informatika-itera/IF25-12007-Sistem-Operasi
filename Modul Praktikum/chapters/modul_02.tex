% ============================================================
%  Modul 2 — Instalasi Ubuntu di VirtualBox
% ============================================================

\chapter{Instalasi Ubuntu di VirtualBox}
\label{chap:modul-02}

% --- 1. Tujuan dan Output Praktikum ---
\section{Tujuan dan Output Praktikum}

\subsection{Tujuan Praktikum}
Setelah menyelesaikan praktikum ini, mahasiswa diharapkan mampu:
\begin{enumerate}
    \item Menjelaskan langkah-langkah instalasi sistem operasi Ubuntu pada lingkungan VirtualBox.
    \item Mengonfigurasi mesin virtual sesuai dengan kebutuhan sistem (alokasi RAM, storage, dan CPU).
    \item Melakukan proses instalasi Ubuntu hingga sistem dapat dijalankan dengan baik sebagai Guest Operating System.
\end{enumerate}

\subsection{Output Praktikum}
Pada akhir praktikum ini, mahasiswa diharapkan menghasilkan:
\begin{itemize}
    \item Mesin virtual Ubuntu yang berhasil terinstal dan dapat dijalankan pada VirtualBox.
    \item Dokumentasi proses instalasi (screenshot tahapan penting instalasi).
    \item Laporan praktikum yang memuat konfigurasi mesin virtual serta hasil verifikasi sistem.
\end{itemize}


% --- 2. Dasar Teori ---
% \newpage
% \section{Dasar Teori}

% \subsection{Sub-Teori Pertama}
% \lipsum[2]

% \subsection{Sub-Teori Kedua}
% \lipsum[3]

% \subsubsection{Sub-Sub-Teori A}
% \lipsum[4]

% \subsubsection{Sub-Sub-Teori B}
% \lipsum[5]

% \subsection{Sub-Teori Ketiga}
% \lipsum[6]

% Tabel \ref{tab:m02-dasar-teori} merangkum hal-hal penting terkait teori di atas.

% \begin{table}[h]
% \caption{Tabel Ringkasan Dasar Teori}
% \label{tab:m02-dasar-teori}
% \centering
% \begin{tabulary}{\textwidth}{|L|L|}
% \hline
% \textbf{Konsep} & \textbf{Penjelasan Singkat} \\ \hline
% Konsep A        & \textit{TODO: isi penjelasan} \\ \hline
% Konsep B        & \textit{TODO: isi penjelasan} \\ \hline
% Konsep C        & \textit{TODO: isi penjelasan} \\ \hline
% Konsep D        & \textit{TODO: isi penjelasan} \\ \hline
% \end{tabulary}
% \end{table}

% --- 3. Alat dan Bahan ---
\section{Alat dan Bahan}
\label{sec:alat-bahan}

\subsection{Perangkat Keras}
Untuk dapat menjalankan instalasi Ubuntu pada lingkungan virtual, diperlukan perangkat keras dengan spesifikasi minimum sebagai berikut:
\begin{itemize}
    \item Laptop atau PC dengan prosesor minimal dual-core.
    \item RAM minimal 4 GB (disarankan 8 GB untuk kinerja optimal).
    \item Ruang penyimpanan kosong minimal 25 GB.

\end{itemize}
Spesifikasi yang lebih tinggi akan meningkatkan performa mesin virtual selama proses instalasi dan penggunaan sistem.

\subsection{Perangkat Lunak}
Perangkat lunak yang diperlukan dalam praktikum ini meliputi:
\begin{itemize}
    \item Sistem Operasi utama (Windows, macOS, atau Linux) sebagai Host OS.
    \item Oracle VM VirtualBox 
    \item File ISO Ubuntu .
\end{itemize}
Pastikan VM VirtualBox dan file ISO Ubuntu telah diunduh sebelum praktikum dimuli untuk menghindari keterlambatan proses instalasi.

% --- 4. Langkah-Langkah Praktikum ---
\newpage
\section{Langkah-Langkah Praktikum}

\begin{enumerate}
    \item \textbf{Instalasi VirtualBox}

    Langkah pertama yang harus dilakukan adalah menginstal software virtualisasi, yaitu Oracle VM VirtualBox.

    Unduh VirtualBox melalui tautan resmi berikut:

    \url{https://www.virtualbox.org/wiki/Downloads}

    \item \textbf{Mengunduh File ISO Ubuntu Desktop}

    Setelah VirtualBox berhasil diinstal, langkah berikutnya adalah menyiapkan sistem operasi yang akan dijalankan di dalam Virtual Machine, yaitu Ubuntu Desktop.

    Unduh file ISO Ubuntu Desktop melalui tautan resmi berikut:

    \url{https://ubuntu.com/download/desktop}

    \item \textbf{Setup Virtual Machine}

    Setelah VirtualBox terinstal dan file ISO Ubuntu sudah diunduh, langkah selanjutnya adalah melakukan setup Virtual Machine di VirtualBox.

    \begin{enumerate}
        \item Buka aplikasi Oracle VM VirtualBox, kemudian klik tombol \textbf{New}.

        \begin{figure}[H]
            \centering
            \includegraphics[width=0.75\textwidth]{Figure/02/gambar1.png}
            \caption{Tampilan Oracle VM VirtualBox saat memilih tombol \textbf{New}.}
            \label{fig:virtualbox-klik-new}
        \end{figure}
        \newpage
        \item Pada bagian \textbf{Name}, isi nama mesin virtual (misalnya \texttt{Ubuntu-Elsa}). Pada bagian \textbf{ISO Image}, pilih file ISO Ubuntu yang sudah diunduh, lalu klik \textbf{Next}. 

        \begin{figure}[H]
            \centering
            \includegraphics[width=0.75\textwidth]{Figure/02/gambar2.png}
            \caption{Tampilan Oracle VM VirtualBox saat memilih nama mesin virtual dan file ISO Ubuntu.}
            \label{fig:virtualbox-name-iso}
        \end{figure}

        \item Pada menu \textbf{Unattended Install}, isi \textbf{Username}, \textbf{Password}, dan \textbf{Hostname} (tanpa spasi), kemudian klik \textbf{Next}. 

        \begin{figure}[H]
            \centering
            \includegraphics[width=0.75\textwidth]{Figure/02/gambar3.png}
            \caption{Tampilan Oracle VM VirtualBox pada tahap \textbf{Unattended Install}.}
            \label{fig:virtualbox-unattended-install}
        \end{figure}

        \item Pada menu \textbf{Hardware}, atur \textbf{Base Memory} minimal 4096 MB dan \textbf{Processor} minimal 2 core (sesuaikan dengan spesifikasi perangkat), lalu klik \textbf{Finish}. 

        \begin{figure}[H]
            \centering
            \includegraphics[width=0.75\textwidth]{Figure/02/gambar4.png}
            \caption{Tampilan pengaturan \textbf{Hardware} pada Oracle VM VirtualBox (Base Memory dan Processor).}
            \label{fig:virtualbox-hardware}
        \end{figure}

        \item Jalankan mesin virtual yang sudah dibuat, lalu pada tampilan awal instalasi pilih opsi \textbf{"Try or Install Ubuntu"}.

        \begin{figure}[H]
            \centering
            \includegraphics[width=0.75\textwidth]{Figure/02/gambar5.png}
            \caption{Tampilan awal instalasi Ubuntu saat memilih opsi \textbf{"Try or Install Ubuntu"}.}
            \label{fig:virtualbox-try-install}
        \end{figure}

        \item Lanjutkan proses instalasi sampai tahap \textit{copying files} selesai. Proses ini biasanya memerlukan waktu beberapa puluh menit, kemudian lakukan restart VM jika diminta. 
\end{enumerate}
\end{enumerate}

\newpage
\section{\textit{Hands On}}

Pada bagian ini, setiap mahasiswa wajib mengerjakan tugas praktik mandiri sebagai berikut:
\begin{enumerate}
    \item Menginstal Ubuntu pada Oracle VM VirtualBox hingga sistem dapat berjalan dengan baik.
    \item Mendokumentasikan setiap langkah instalasi secara berurutan (disertai bukti seperti tangkapan layar).
    \item Menyusun laporan praktikum sesuai template resmi pada GitHub IF ITERA:
    \url{https://github.com/informatika-itera/IF25-12007-Sistem-Operasi/tree/main/Hands-On}
\end{enumerate}

Laporan \textbf{wajib} disusun menggunakan \LaTeX.
